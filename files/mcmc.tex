\documentclass[angelino.tex]{subfiles} 
\begin{document}

\emph{Markov chain Monte Carlo} (MCMC) is a widely used, powerful technique for 
estimating statistics of an arbitrary distribution~$\Pi$
%with probability density function~$\pi(x)$,
defined over a state space~$\X$.
MCMC simulates a random walk that produces a sequence of samples drawn from a 
sequence of distributions that converges to~$\Pi$.
% with densities that converge to~$\pi(x)$.
MCMC is typically employed when samples from, or statistics of,
a distribution cannot be obtained analytically, as is often the case
with complex, high-dimensional systems arising across disciplines,
\eg estimating bulk material properties from molecular dynamics physics
simulations or inferring the parameters of Bayesian probabilistic models
describing large datasets.

In this chapter, we first review the two powerful tools underlying
MCMC algorithms -- \emph{Markov chains} and \emph{Monte Carlo} methods.\eanote{Swap Markov chain and Monte Carlo sections?}
Next, we introduce MCMC via the well-known \emph{Metropolis-Hastings} algorithm,
both as a way to concretely exemplify relevant concepts and to motivate a large
body of research whose goal is to design more efficient MCMC algorithms.
We then provide an overview of different classes of these approaches, with
greater focus on the areas that together provide the foundation for a new
approach to large-scale MCMC that we present in the next chapter.

\section{Markov chains}

Let~$\X$ be a discrete or continuous state space and
let~$x, x' \in \X$ denote states.
A Markov chain
%$\Phi = \{\phi_0, \phi_1, \dots, \}$ is a countable set of
%random variables.
is a discrete-time stochastic process governed by a
transition operator~${T(x \rightarrow x')}$ 
that specifies the probability of transitioning
from a current state~$x$ to some next state~$x'$.
It is \emph{memoryless} in the sense that its future behavior depends only
on the current state and is independent of its past history --
this is known as the Markov property.
%Markov chains were developed by Andrey Markov in the early twentieth century.

Many systems can be modeled by Markov chains.
For example, an unbiased random walk on a one-dimensional lattice is described
by a Markov chain.  The integers modulo~$k$ can be used to index a finite
lattice of~$k$ states, in which case~${\X = \ints_k}$.
The transition operator,
\be
T(x \rightarrow x-1 \mod k) = T(x \rightarrow x+1 \mod k) =  T(x \rightarrow x \mod k) = \frac{1}{3}, \label{random-walk}
\ee
describes a random walk on the lattice, with periodic boundaries,
that at each time step either moves to the `left' or `right' by one unit,
or stays put, where the three scenarios are equiprobable.
%\eanote{This random walk or diffusive behavior is slow at
%exploring the space, requiring in expectation $O(k^2)$ time steps for an
%instance of the  Markov chain to reach a distance $d = k$ units away from any
%initial condition, \ie visit all states. We can see this by observing \dots}
Here, the stationary distribution is simply the uniform distribution over
$\ints_k$.

Given an initial distribution~$P^0(x)$ over~$\X$, a Markov chain evolves this
distribution from one time point to the next through iterative application of
the transition operator; after~$t$ steps let us call this distribution~$P^t(x)$.
Direct \emph{simulation} of a Markov chain follows this iterative construction
and leads to inherently serial implementations.
We are interested in Markov chains that \emph{converge} to a unique
\emph{stationary distribution}~$\pi(x)$ in the sense that
\[
\lim_{t \rightarrow \infty} P^t(x) \rightarrow \pi(x),
\]
for any initial distribution~$P^0(x)$.

The \emph{speed of convergence} or \emph{mixing time} of a Markov chain measures
how quickly~$P^t(x)$ approaches~$\pi(x)$; it is typically defined with respect to
a distance measure between probability distributions and a threshold.
For example, it could be defined as the minimum or expected number of steps~$t$
such that~${D_{\text{KL}}(\pi(x) ~ \| ~ P^t(x)) < \epsilon}$,
for some appropriate~${\epsilon > 0}$, where $D_{\text{KL}}(P \| Q)$
is the \emph{Kullback-–Leibler divergence} of two distributions~$P$ and~$Q$,
and we think of~$Q$ as an approximation to~$P$ \citep{kullback:1951}.
Convergence behavior depends on the properties of the state space~$\X$
-- \eg whether it is discrete or continuous, its dimensionality --
and the behavior of the transition operator.
%Suppose two Markov chains can have the same stationary distribution
%but one mixes faster.
%Intuitively, this occurs because its transition operator encourages larger
%`jumps' between states, \eg by having a smaller probability of self-transition
%$T(x \rightarrow x)$ for all $x \in \X$.

For example, consider a simulation of a one-dimensional, $k$-state random walk,
described by the transition operator in Eq.~\ref{random-walk}.
The mixing time is~$O(k^2)$, \ie the simulation requires~$O(k^2)$ steps to
`forget' the initial condition and look reasonably like
the uniform stationary distribution.
\eanote{Explain.}
In contrast, consider a deterministic transition operator that always moves
to the `right', \ie $T(x \rightarrow x + 1) = 1$.
This time, simulation requires only~$O(k)$ steps to approach
the uniform stationary distribution.
While this simple example represents an extreme case that is not useful for
typical applications, it illustrates how two Markov chains can have the
same stationary distribution but different convergence behavior.
A major area of Markov chain research is understanding how to design efficient
transition operators that converge quickly, as doing so has direct practical
consequences for their simulation.

For a transition operator~$T(x \rightarrow x')$ to have~$\pi(x)$ as its
stationary distribution, its application must leave~$\pi(x)$ invariant
over the entire space, \ie
\be
\sum_{x \in \X} T(x \rightarrow x') \pi(x) = \pi(x'), \quad \forall x' \in \X \nn
\ee
for a discrete state space, or
\be
\int_\X T(x \rightarrow x') \pi(x) dx = \pi(x'), \quad \forall x' \in \X \label{stationary}
\ee
for a continuous state space; this thesis will focus on continuous state spaces.
For the stationary distribution to be unique, \ie not depend on
the initial distribution, the Markov chain must be \emph{irreducible}:
for any $x, x' \in \X$ such that $\pi(x), \pi(x') > 0$,
it must be possible to reach $x'$ from $x$ in a finite number of steps.
A powerful application of Markov chains involves designing a transition operator
that has as its stationary distribution some target distribution of interest
-- this is the main idea behind Markov chain Monte Carlo methods.

In restricted cases it is easy to show that a transition operator has a certain
stationary distribution.
Notably, when a transition operator $T(x \rightarrow x')$ is \emph{reversible},
it satisfies \emph{detailed balance} with respect to a distribution~$\pi(x)$,
\be
T(x \rightarrow x') \pi(x) = T(x' \rightarrow x) \pi(x'), \label{detailed-balance}
\ee
and it is easy to show that~$\pi(x)$ is its stationary distribution.
Integrating over~$\X$ on both sides gives:
\bea
\int_\X T(x \rightarrow x') \pi(x) dx
&=& \int_\X T(x' \rightarrow x) \pi(x') dx \nn \\
&=& \pi(x') \int_\X T(x' \rightarrow x) dx \nn \\
&=& \pi(x'), \nn
\eea
which is precisely the required condition from Eq.~\ref{stationary}.
We can interpret Eq.~\ref{detailed-balance} as stating that,
for a reversible Markov chain starting from its stationary distribution,
any transition $x \rightarrow x'$ is equilibrated by
the corresponding reverse transition $x' \rightarrow x$.
As we will see, many MCMC methods are based on deriving reversible
transition operators.
A transition operator that is not reversible is called \emph{non-reversible};
it is generally more difficult to manipulate and prove statements about these.

%Finally, we note that since the stationary distribution of a Markov chain is
%rarely known \emph{a priori}, a practical challenge of simulating Markov chains
%is that it is difficult to assess whether a materialized chain has effectively
%converged; this will be one of our concerns in our empirical studies. 

For a formal introduction to Markov chains, see the book by~\citet{meyn-tweedie:1993}.

\section{Monte Carlo methods}

%Monte Carlo methods as we know them were developed in the 1940s
%by Stanislaw Ulam and John von Neumann at at Los Alamos Scientific Laboratory.
Monte Carlo methods are a broad class of algorithms that simulate many repeated
random samples to estimate some quantity of interest.
%Molecular dynamics simulations, blah and blah are all Monte Carlo simulations.
For example, the following procedure is a form of
\emph{Monte Carlo integration} that estimates the area under any positive
function ${f: [a, b] \rightarrow \reals^+}$,
where ${-\infty < a < b < \infty}$:
\begin{enumerate}
\item Draw a box around~$f$ with vertical boundaries set by the
interval~$[a, b]$ and horizontal boundaries set by~$0$
and an upper bound~$m$ on the maximum value of~$f$ in the interval.
\item Sample a large number of random points~$(x, y)$ uniformly
within the box and for each, determine whether the point falls below or
above~$f$ by computing whether $f(x) < y$.
\item Let $r$ be the fraction of points such that~$f(x) < y$.
Since the total area of the box is $m(b-a)$, multiplying by~$r$
provides an estimate for $\int_a^b f(x) dx$.
\end{enumerate}

More generally, when we can think of an integral as an expectation,
Monte Carlo integration invokes the law of large numbers
to estimate this expectation via a sample average.
Specifically, if we can write an integral as the expectation of a
function~$f(x)$ with respect to a distribution~$\Pi$ with probability density
function $\pi(x)$,
\be
\E_\Pi(f) = \int f(x) \pi(x) dx,
\label{expectation}
\ee
then we can estimate this integral by averaging over a set of 
samples~$\{x_n\}_{n=1}^N$ from~$\Pi$ as:
\[
\bar{f}_N \equiv \frac{1}{N} \sum_{n=1}^N f(x_n).
\]
Since the samples are independent, as long as the expectation in 
Eq.~\ref{expectation} exists and is finite,
this sum obeys the law of large numbers.
Hence, the estimate is unbiased and its variance scales as the inverse sample
size $1/N$, or equivalently, its error scales as $1/\sqrt{N}$.
In our example above, the integral of $f(x)$ on the interval $[a, b]$ can be
thought of as an expectation with respect to the uniform distribution on
$[a, b]$.
%More concretely, we can write down a central limit theorem (CLT) for $f_n$:
%\be
%\lim_{N \rightarrow \infty)
%\sqrt{N}(\bar{f}_N - \E_\Pi(f)) \rightarrow \N(0, \sigma^2),
%\ee
%where the variance is bounded and given by
%\be
%\var^2 = \Var_\Pi(f(X_0)) + 2 \sum_{n=1}^\infty \Cov_\Pi(f(X_0, f(X_i)).
%\ee

Monte Carlo integration thus requires sampling from a distribution, which is
sometimes straightforward, as with the uniform and normal distributions,
but in general requires numerical simulation.
Below, we describe two additional Monte Carlo methods that address this issue
in restricted settings:
\emph{rejection sampling} and \emph{importance sampling}.
Their limitations and inefficiencies will help motivate Markov chain
Monte Carlo methods, which are more sophisticated but related techniques.
For simplicity, we describe these procedures with respect to one-dimensional
normalized probability densities; both can be generalized.

\subsection{Rejection sampling}

Rejection sampling uses one distribution to sample from another
by exploiting information relating the two;
\citet{von-Neumann:1951-random} provided an algorithm for this method.
Suppose that we want to sample from a \emph{target distribution}~$\Pi$ with
probability density function~$\pi(x)$.
Suppose further that we can sample from a \emph{proposal distribution}~$Q$
whose probability density function~$q(x)$ can be scaled by a
constant factor~$\gamma$ to provide an upper bound on~$\pi(x)$,
\eg we might be able to scale a normal distribution so that our distribution of
interest lies below it everywhere.
If we satisfy these requirements, then we can use
rejection sampling to generate \emph{proposals} from~$Q$ that we stochastically
\emph{accept} or \emph{reject} according to the relative difference
between~$\gamma q(x)$ and~$\pi(x)$. Specifically, to produce one sample:
\begin{enumerate}
\item Generate a proposal $x$ by drawing a sample from the
      proposal distribution~$Q$.
\item Draw a sample $y$ uniformly from the interval $[0, \gamma q(x)]$.
\item If $y < \pi(x)$, accept~$x$.
      Otherwise, reject~$x$ and return to Step 1.
\end{enumerate}
Rejection sampling is most efficient in the limit where the scaled proposal
density equals the target density, in which case all proposals are accepted.
More generally, in expectation, this procedure accepts proposals at a rate given
by~$\int \pi(x) / (\gamma q(x)) dx \le 1$.

\subsection{Importance sampling}

Similar to rejection sampling, importance sampling also uses information from
one distribution to sample from another, but with fewer restrictions.
Suppose we have distributions~$\Pi$ and~$Q$ as above, where this time we
can simply think of~$q(x)$ as an approximation to~$\pi(x)$;
\ie we do not require some~$\gamma q(x)$ that is an upper bound to~$\pi(x)$.
Suppose we want to compute the expectation of some function~$f(x)$ with
respect to the distribution~$\Pi$:
\[
\E_\Pi (f(x)) = \int f(x) \pi(x) dx.
\]
By multiplying and dividing by~$q(x)$ inside the integral,
\[
\E_\Pi (f(x)) = \int \frac{f(x) \pi(x)}{q(x)} q(x) dx
\equiv \E_Q \left(f(x) w(x) \right),
\]
we change nothing, but can interpret this new expression as the
expectation of $f(x)$ weighted by $w(x) = p(x) / q(x)$ with respect to $Q$.
We can Monte Carlo estimate this integral using a set of samples
$\{x_n\}_{n=1}^N$ from $Q$:
\[
\frac{1}{N} \sum_{n=1}^N f(x_n) w(x_n).
\]
%The $w(x^n)$ are called \emph{importance ratios} or \emph{importance weights}.
The quality of this estimator depends on how much $f(x) w(x)$ varies --
ideally this quantity would be constant with respect to $x$.
Some historical notes and a list of references on importance sampling can be
found in the textbook by~\citet{gelman:1993-bda}.

\subsection{Limitations of Monte Carlo sampling}

The primary limitation of both rejection sampling and importance sampling is
that for these methods to be feasible and practical, each requires a proposal
distribution that can be sampled easily and is in some sense
close to the target distribution.
To produce samples, both methods use a set of
independent samples from the proposal distribution;
rejection sampling selects from among the proposals
and importance sampling `fixes up' the proposals by assigning each a weight.
\todo{Ex. where IS fails: var(pi/q) huge, estimate meaningless.}

\section{Markov chain Monte Carlo}

Markov chain Monte Carlo (MCMC) methods simulate a Markov chain whose stationary
distribution is equal to a target distribution of interest.
When this Markov chain is simulated, it produces samples from a sequence of
distributions that asymptotically equals the target distribution.
The principles of Monte Carlo integration, estimation and sampling thus apply to 
these samples in the asymptotic limit.
Concretely, for a Markov chain started from its stationary distribution~$\Pi$
with density~$\pi(x)$, a sequence of $N$ samples $\{x_n\}_{n=1}^N$ can be used
to estimate an expectation~$E_\Pi(f) = \int_X f(x) \pi(x) dx$ using Monte Carlo
integration via the sample average~$\bar{f}_N = \frac{1}{N} \sum_{n=1}^N f(x_n)$.
The efficiency of a MCMC transition operator can be analyzed with respect to
both the variance of this estimator, also known as the
\emph{asymptotic variance}, as well as the speed of convergence or mixing time.
In practice, we use samples produced by simulated chains of finite length,
typically started away from stationarity.
The materialized sequence of samples obeys the Markov property and is
correlated, which is in contrast to the independent samples obtained by simple
Monte Carlo methods such as rejection sampling and importance sampling.

The remaining sections of this chapter give an incomplete overview of
MCMC algorithms for sampling applications, with greater emphasis on
certain procedures either for the purpose of providing general background
or to review those most directly related to this thesis.
%We do not discuss other important applications of MCMC methods,
%such as optimization.
First, we describe the Metropolis-Hastings (MH) algorithm, a canonical and
simple MCMC method.
We use MH to build some intuition for the behavior of MCMC algorithms,
and to illustrate its limitations that motivate more sophisticated approaches.
The following two sections classify these further approaches into serial
algorithms designed to converge more quickly than MH and parallel algorithms.
Finally, we briefly review MCMC algorithms that exploit common features of
Bayesian inference problems.
For a general introduction to MCMC, see the highly motivating review by~\citet{diaconis-2008-revolution}.

\section{Metropolis-Hastings (MH)}

%\eanote{Metropolis, Metrolis-Hastings, Gibbs}

%\noi Consider any system that can be described by a fixed probability density
%distribution $p$ over some state space $\X$.
%Suppose we are interested in generating samples according to distribution $p$
%or computing statistics of this distribution.
%While these tasks can be performed exactly and analytically for some
%distributions, they are in general intractable.

% hydrogen bomb...
% In 1953, again at Los Alamos Scientific, Metropolis \etal were interested in
% quantities computed from the equilibrium distribution of a system of $N$
% particles governed by their pairwise interactions~\cite{metropolis-1953}.
% Here, the state space $\X$ is $$
% is the set of all possible particle configurations
% on a lattice and the target distribution of interest $p$ .

The Metropolis-Hastings (MH) algorithm simulates a Markov chain, over a
state space~$\X$, with stationary distribution equal to some target
distribution of interest.
Given an initial state $x_0$, a target distribution~$\pi$ and
a proposal function~$q(x' \vert x)$,
MH generates a sequence of states~$x_1, \dots, x_T \in \X$
drawn from a sequence of distributions that converges to the target.\footnote{
As is common in the literature, we will henceforth use the same symbol
to refer to both a distribution and its probability density function;
the interpretation should be clear from context.}
We provide pseudocode for MH in Algorithm~\ref{mh}.
Each iteration, a proposal for the next state~$x'$ is drawn from the
proposal distribution, conditioned on the current state~$x$;
\eg a common choice is to sample from a Gaussian centered at~$x$.
The proposal is stochastically accepted with probability given by the 
\emph{acceptance~ratio},
\be
r = \frac{\pi(x') q(x \given x')}{\pi(x) q(x' \given x)},
\ee
via comparison to a random variate~$u$ drawn uniformly from the interval~$[0, 1]$.
If~${u < r}$, then the next state is set to the proposal, otherwise,
the proposal is rejected and the next state is set to the current state.
MH is a generalization of the \emph{Metropolis algorithm}~\citep{metropolis-1953}, 
which requires the proposal distribution to be symmetric,
\ie~${q(x' \vert x) = q(x \vert x')}$, in which case
the acceptance ratio is simply~${r = \pi(x') / \pi(x)}$.
\citet{hastings-1970} later relaxed this by showing that the proposal
distribution could be arbitrary.

%The canonical MCMC algorithm is Metropolis-Hastings, in which generating
%each sample requires evaluating the posterior at a proposed value of $\theta$.
%Evaluating the likelihood term is the dominant and expensive computational cost,
%since it is a function of the entire input dataset.

\begin{algorithm}[t]
\caption{Metropolis-Hastings}
\label{mh}
\begin{algorithmic}
\State \textbf{Input:} Initial state $x_0$, number of iterations $T$, target $\pi(x)$, proposal $q(x' \given x)$
\State \textbf{Output:} Samples $x_1, \dots, x_T$
\For {$t$ in $0, \dots, T-1$}
\State $x' \sim q(x' \given x_t)$ \Comment{Generate proposal}
\State $r \gets \dfrac{\pi(x') q(x_t \given x')}{\pi(x_t) q(x' \given x_t)}$
       \Comment{Compute acceptance ratio}
\State $u \sim \Unif(0, 1)$ \Comment{Draw random number}
\If {$u < r$}
    \State $x_{t+1} \gets x'$ \Comment{Accept proposal}
\Else
    \State $x_{t+1} \gets x_t$ \Comment{Reject proposal}
\EndIf
\EndFor
\end{algorithmic}
\end{algorithm}

The MH algorithm can be viewed as a biased random walk that
always accepts proposals when
${\pi(x') q(x \given x') > \pi(x) q(x' \given x)}$ and
stochastically rejects them otherwise;
for a symmetric proposal distribution, these scenarios can be
interpreted as accepting `uphill' proposals and
stochastically rejecting `downhill' proposals.
We can see that the stationary distribution is indeed~$\pi$ by showing
that the MH transition operator satisfies detailed balance,
as defined in Eq.~\ref{detailed-balance}.
From the algorithm description, the MH transition operator is:
\[
T(x \rightarrow x') = \min(1, r) q(x' \given x) =
\min\left(1, \frac{\pi(x') q(x \given x')}{\pi(x) q(x' \given x)}\right) q(x' \given x).
\]
We can verify the detailed balance condition as follows:
\bea
T(x \rightarrow x') \pi(x)
&=& \min\left(1, \frac{\pi(x') q(x \given x')}{\pi(x) q(x' \given x)}\right) q(x' \given x) \pi(x) \nn \\
&=& \min\left(\pi(x) q(x' \given x), \pi(x') q(x \given x')\right) \nn \\
&=& \min\left(\frac{\pi(x) q(x' \given x)}{\pi(x') q(x \given x')}, 1\right) q(x \given x') \pi(x') \nn \\
&=& T(x' \rightarrow x) \pi(x'). \nn
\eea

\subsection{Factors affecting the behavior of MH}
\label{sec:mh-behavior}

The MH algorithm is both simple to implement and quite general;
it is thus appealing and widely applicable.
However, the MH algorithm has a major drawback -- it can be slow to converge.
This is due to the fact that the steps of the underlying Markov chain
are correlated, which can be viewed as random walk or diffusive behavior.
One broad strategy for increasing the efficiency of MCMC methods is to
design transition operators that behave less like simple diffusion;
we survey several techniques for doing so in the next section.

Within the MH framework and given a target density, the variable parameters are
the proposal distribution and the initial condition.
Let us first consider the proposal distribution.
For example, for a one-dimensional continuous target density, if we restrict
the proposal distribution to be Gaussian and centered at the current state,
$q(x' \given x) = \N(x' \given x, \sigma^2)$,
then there is a single tuning parameter: the distribution's standard
deviation~$\sigma$, which gives the expected `step size' of the proposal
with respect to the current state.
This affects the MH \emph{acceptance rate}, 
which we also refer to as the \emph{acceptance probability},
\ie the fraction of proposals that are accepted.

To illustrate the relationship between the proposal distribution and the
acceptance rate, consider unimodal target and proposal distributions.
Suppose that we are able to initialize MH at a state
close to the target distribution's mode with respect to its width.
Intuitively, if the proposal step size is large compared to the width of the
target, then proposals will tend to fall in faraway, low-probability regions,
resulting in a low acceptance rate.
On the other hand, if the step size is very small, then the target density
at the proposal will be very close to that at the current state,
in which case the algorithm will tend to accept proposals,
but the samples will be highly correlated and the chain will take a long time
explore the area under the target density.
This suggests that there is some notion of an optimal MH acceptance rate
corresponding to some intermediate proposal step size.

A classic result is that the optimal value for the MH acceptance rate is~0.234, 
derived for the scenario where the target and proposal distributions are 
multidimensional Gaussians, in the limits where the chain has converged and
the number of dimensions tends to infinity~\citep{roberts-1997-accept}.
A heuristic widely followed by practitioners is to tune the proposal
distribution to obtain an observed acceptance rate of about~0.234.

The sensitivity of the acceptance rate as a function of
the proposal distribution also explains why the MH algorithm
has trouble sampling from multimodal target densities.
When modes are far apart compared to the widths of the peaks around them,
they are separated by low-probability regions that are difficult for a
simulated MH chain to traverse.
In these cases, the MH algorithm tends to get `stuck' for many iterations
around local modes, instead of sampling globally from the entire distribution.
In practice, a MH simulation tends to find the mode closest to the initial state
and then samples the area around this mode.

Given target and proposal distributions, the only other specification required
by the MH algorithm is an initial state.
Clearly, it is desirable for the initial state to be close to some probable
region of the target density -- 
a `bad' initial state combined with the random walk nature of chain simulation
yields initial samples that are not representative of the target.
This initial portion of a MCMC simulation, before convergence,
is sometimes called \emph{burn-in}.

The behavior of a MCMC simulation during burn-in is different from that after
convergence, because the shape of the target density differs
far from versus close to the bulk of its mass.
Specifically, the target density tends to be `flatter' or `less steep'
around a mode compared to less probable regions.
This characterization interacts with proposal generation,
resulting in acceptance behavior that changes from burn-in to convergence.

To illustrate differences in MCMC behavior between burn-in and convergence, 
consider MH for a Gaussian target distribution.
\todo{Picture.}
Typically, a MCMC simulation is initiated at some informed guess that is still
somewhat far from higher probability regions of the target; assuming it is
well-behaved, the chain should eventually spend more time in these regions.
A Gaussian distribution has its mass concentrated around a single mode.
A region close to this mode can be well-approximated by an upside down parabola
-- a quadratic function -- while the tails fall off exponentially quickly.
Suppose also that our proposal distribution is symmetric and its width
is not large compared to the width of the target.
In the region close to the target mode, the target densities
evaluated at two nearby states will tend to be comparable values.
In the context of MH, the acceptance ratio~$r$ will be well within the
interval~$[0, 1]$ and the decision to accept or reject a proposal depends on
both~$r$ and the random variate~$u$.
If we consider two nearby states in the tail regions, then the target
density evaluated at one will be exponentially smaller than the other.
Here, the acceptance ratio~$r$ will be close to either 0 or 1, so the
random variate~$u$ has little influence over whether a proposal is accepted.
As we will see later, these differences between chains during burn-in and
convergence have implications for the performance of our new approach to
MCMC as well as our empirical studies.

% Adaptive MH.

%In practice, MCMC is used in scenarios where the target distribution $p$ is
%unknown and so it can be difficult or impossible to be certain whether the
%algorithm has converged and is producing samples representative of $p$.
%There are several common heuristics, without any particular 

%\subsection{Gibbs sampling}

\section{MCMC methods for faster convergence}

In this section, we survey classes of MCMC algorithms designed to
converge more quickly than the MH algorithm by reducing the 
correlation between successive states.
We do not provide a thorough review, as the methods we develop in this
thesis do not build directly on these techniques.
However, we do describe specific algorithms both for concreteness, and because
we will later consider them within the context of \emph{predictive prefetching},
a new framework we present in Chapter~\ref{sec:prefetching}.

%Below, we review two broad classes of approaches to speeding up MCMC:
%designing algorithms that discourage diffusive behavior
%in order to decrease the mixing time and parallel approaches.

\subsection{Auxiliary variable methods}

Given a target density~$\pi(x)$, we can introduce an
\emph{auxiliary variable}~$y$ and define a new density~$\pi(x, y)$
such that~$\int \pi(x, y) dy = \pi(x)$,
\ie marginalizing out $y$ the yields the target.
Auxiliary variable methods design MCMC sampling schemes over the space of a new
joint distribution; after sampling from~$\pi(x, y)$, one obtains samples 
from~$\pi(x)$ simply by ignoring the~$y$ values.
While it would seem less desirable to sample from a higher dimensional space,
it is possible to design transition operators over the joint space that
marginally sample from the target in a way that is more efficient than
Metropolis-Hastings.

For example, consider a one-dimensional target density
$\pi(x): \reals \rightarrow \reals^+$.
Sampling from~$\pi(x)$ yields a sequence of samples along the real line.
Now consider a representation of the target in the $(x, y)$-plane
such that $y = \pi(x)$.
If we sample a set of points $\{(x_i, y_i)\}$ uniformly within the 
two-dimensional area between~$\pi(x)$ and the $x$-axis, then marginally,
the~$\{x_i\}$ are samples from~$\pi(x)$.
Below, we summarize two auxiliary variable methods:
slice sampling and Hamiltonian Monte Carlo.

\emph{Slice sampling} methods are based on the above idea, sampling from the
joint distribution~$\pi(x, y)$ by iteratively sampling each variable 
marginally~\citep{neal:2003-slice}.
Given some state~$x_i$, the procedure constructs~$y_i$ and then
the next~$x_{i+1}$ as follows:
\begin{enumerate}
\item Sample $y_i \sim \pi(y_i \given x_i)$ by sampling
      uniformly from the (vertical) interval $[0, \pi(x_i)]$.
\item Sample $x_{i+1} \sim \pi(x_{i+1} \given y_i)$ by sampling uniformly
      from the (horizontal) intervals where $\pi(x) > y_i$.
\end{enumerate}
We think of $y_i$ as defining a horizontal `slice' through the distribution.
Slice sampling has multiple advantages over Metropolis-Hastings.
The procedure has the opportunity to mix well with respect to sampling from
the target distribution, because a horizontal slice may correspond to a
large domain that is sampled uniformly, so $x_{i+1}$ can be very far from $x_i$.
In practice, it can be tricky to sample the~$x_i$ since doing so in full
would require constructing the inverse of $\pi(x)$, but there are various
procedures for avoiding this issue while maintaining correctness.
Notice also that there is no proposal distribution in slice sampling,
which means fewer tuning parameters. % and no notion of rejected proposals.

\emph{Hybrid Monte Carlo} (HMC) introduces an auxiliary `momentum' variable
to embed the action of sampling from the target density~$\pi(x)$ within a
physical system  described by classical mechanics~\citep{duane:1987-hmc};
it is also called \emph{Hamiltonian Monte Carlo}~\citep{neal:2010-hmc}.
First, think of~$(x, -\pi(x))$ as defining an `upside down' surface where the
original modes of~$\pi(x)$ are `valleys' and low-probability regions are
`uphill.'
Now consider a frictionless puck with mass $m$ moving around this surface --
its dynamics will be described by its position and its momentum.
HMC generates a proposal for a Metropolis algorithm by giving the puck a kick
in some direction with some velocity, both random.
The puck's trajectory is simulated for some fixed amount of time $\tau$ by
integrating the system's equations of motion; the final position at time $\tau$
is the proposal.
This can generate faraway but useful proposals because the puck
tends to go downhill toward high-probability regions;
it glides over flat equiprobable regions and loses momentum by moving
uphill toward low-probability regions.

%Swendsen-Wang \\

\subsection{Ensemble methods}
\label{sec:ensemble}

\emph{Ensemble} (or \emph{population}) methods run multiple chains and
accelerate mixing by sharing information between the chains.
Examples include affine-invariant ensemble
sampling~\citep{goodman-2010-ensemble} and
generalized elliptical slice sampling~\citep{nishihara-2014-gess}.
Below, we focus on a class of ensemble approaches known as
\emph{annealing methods}.

Recall that the MH algorithm has trouble sampling from multimodal distributions.
Informally, `flatter' distributions are easier to sample from compared to
`peaky' distributions, especially multimodal ones.
Now consider the probabilistic interpretation of a physical
multi-state system at temperature $\tau > 0$: for a state~$x \in \X$,
its probability~$p(x)$ is proportional to the exponential of the negative of
its energy~$E(x)$ divided by the temperature, \ie
\be
p(x) \propto \exp(-E(x) / \tau). \label{energy}
\ee
For a given system defined by states and their energies, raising the temperature
has the effect of flattening the distribution over those states, while
maintaining important features.
Annealing methods leverage this intuition to sample
more efficiently from difficult targets.

%The idea of annealing comes from the popular optimization heuristic of
%\emph{simulated annealing} (SA).
%While it is not a MCMC method, we review SA for historical reasons and because
%it is easily understood as a modification to the Metropolis-Hastings algorithm.

As an example of a popular annealing method, we describe
\emph{parallel tempering}~\citep{iba:2001-ensemble}.\footnote{Following~\citet{murray-2007-thesis}, we cite a review that chronicles
the history of parallel tempering.}
Let~$\pi(x)$ be the target density over a state space~$\X$.
The idea is to construct a single Markov chain on the product space~$\X^K$
corresponding to an \emph{ensemble} of~$K$ Metropolis-Hastings
simulations of the system specified by~$\pi(x)$ and Eq.~\ref{energy} or its 
continuous analog, each at a different temperature.
Simulations at higher temperatures explore the space more quickly than those
at lower temperatures, and they can share information through interactions.
One of the~$K$ simulations is constructed to marginally have as its stationary 
distribution the target~$\pi(x)$.
Explicitly, we can define the system via an energy function of the
form~${\mathcal{E}(x) = -\log(\pi(x))}$.
Now we specify~$K$ distributions:
\[
\pi_k(x) \propto \exp(-\mathcal{E}(x) c_k), \quad k = 1, \dots K,
\]
where $c_k$ can be interpreted as an inverse temperature.
Notice that~$c_k = 1$ yields~$\pi_k(x)$ equal to the target~$\pi(x)$,
and~$c_k = 0$ results in a constant.
Thus to obtain~$K$ copies of the system, with one equal to the target and
the rest at higher temperatures, we can choose the~$c_k$ so that
${c_1 = 1 > c_2 > c_3 > \dots > c_K \ge 0}$.
In each iteration of the algorithm, the~$K$ simulations are advanced according
to a MH acceptance rule, but they are also allowed to interact,
\eg a pair of simulations may exchange states.
Thus, the slower mixing chain indexed by~$k=1$ may jump to
states explored by faster mixing chains at higher temperatures.
Parallel tempering is popular because its implementation is a straightforward
modification to the MH algorithm.

There are several additional classes of annealing methods and
other ensemble methods; an excellent review can be found in the PhD thesis 
by~\citet{murray-2007-thesis}.

%simulated tempering / expanded ensembles
%annealed importance sampling
%tempered transitions

%\subsection{Particle methods}

\subsection{Non-reversible methods}

The methods described above are representative of the rich menagerie of MCMC algorithms 
developed using reversible Markov chains where the probability that the chain
is in state~$x$ and transitions to state~$x'$ is equal to the probability that
it is in state~$x'$ and transitions to~$x$.
This condition of detailed balance is straightforward to check, which helps
explain the invention of many reversible MCMC methods.
Recall that the goal of these methods is to discourage the
diffusive behavior of simple Metropolis-Hastings.
Intuitively, diffusion is not an efficient mechanism for mixing, say, a cake
batter -- one uses a spoon or electric mixer to induce a flow that is not
equilibrated by a flow in the opposite direction.

Such non-reversibility that discourages `backtracking' has been difficult to
study; a handful of articles describe methods limited to discrete state spaces.
These include the theoretical and numerical analysis 
by~\citet{diaconis:2000-non-reversible} of a simple non-reversible chain.
The authors start with a reversible unbiased random walk on a one-dimensional
finite lattice and then make two copies of the state space,
one `upstairs' for transitions to the `right' and
one `downstairs' for transitions to the `left',
plus transitions between the two levels.
This non-reversible chain converges more quickly
according to two different distance metrics.
\citet{geyer:2000-non-reversible} reanalyze the same system, this time with
respect to asymptotic variance, and find that the most efficient version of the 
non-reversible chain sweeps through the states in a deterministic way.
In a related fashion, \citet{neal:2004-non-reversible} constructs non-reversible 
chains from reversible chains and demonstrates that their asymptotic variance is
no worse than the original reversible chains.
Other non-reversible schemes are inspired by non-diffusive physical systems,
such as a method for inserting `vortices' by~\citet{sun:2010-vortices}.

%\cite{diaconis:2013-things}
%\cite{turitsyn:2008-irreversible}

\section{Parallel MCMC}

The most obvious way to parallelize MCMC is to run independent simulations in
parallel and aggregate their samples.
However, this embarrassingly parallel approach does not help to reduce the
mixing time, which can be prohibitively long and would be replicated across
the parallel instances.

In MCMC, the computational cost is most often determined by the expense of
evaluating the target density relative to the mixing time.
For example in Metropolis--Hastings, this cost is incurred when the target is 
evaluated to determine the acceptance ratio of a proposed move.
%in slice sampling~\citep{Neal03}
%an expensive target slows both bracket expansion and contraction.
We focus on the increasingly common case where the target is expensive and the 
dominant computational cost.
This evaluation can sometimes be parallelized directly,
\eg when the target function is a product of many individually expensive terms.
This sometimes arises in Bayesian inference if the target can be easily
decomposed into one likelihood term for each data item.
Scalability (\ie practically achievable speedup) in this setting is limited by
the communication and computational costs associated with aggregating
the partial evaluations.
In general, the target function cannot be parallelized;
we divide methods that accelerate MCMC via other sources of parallelism into
two classes: parallel ensemble sampling and~prefetching.

\subsection{Parallel ensemble samplers}

The ensemble methods discussed earlier run multiple chains that can be simulated
in parallel, where any information sharing between chains requires communication.
Examples include parallel tempering, described in Section~\ref{sec:ensemble},
the emcee implementation~\citep{goodman-2012-emcee}
of affine-invariant ensemble sampling~\citep{goodman-2010-ensemble}
and a parallel implementation of
generalized elliptical slice sampling~\citep{nishihara-2014-gess}.

\subsection{Prefetching}
\label{sec:mcmc-prefetching}

The second class of parallel MCMC algorithms uses parallelism through
speculative execution to accelerate individual chains.
This idea is called \emph{prefetching} in some of the literature and appears to
have received only limited attention.
To the best of our knowledge, prefetching has only been studied in the context
of the MH algorithm where, at each iteration, a single new proposal is drawn
from a proposal distribution and stochastically accepted or rejected.
As shown in Algorithm~\ref{mh}, the body of a MH implementation is a
loop containing a single conditional statement and two associated branches.
We can thus view the possible execution paths as a binary tree,
illustrated in Figure~\ref{tree}.
The vanilla version of prefetching speculatively evaluates all paths in this
binary tree~\citep{brockwell-2006-prefetching}.
The correct path will be exactly one of these, so with~$J$ cores, this approach 
achieves a speedup of~$\log_2 J$ with respect to single core execution,
ignoring communication and bookkeeping overheads.

\begin{figure}[t!]
  \centering%
  \resizebox{\columnwidth}{!}{%
    \def\radius {5mm}
    \tikzstyle{state}=[circle, thick, minimum size=\radius, font=\footnotesize]
    \begin{tikzpicture}[->,>=stealth',level/.style={sibling distance = 5cm/#1, level distance = 1.5cm}]
      \draw [white,-] (-6cm,-0.75cm) -- (6cm,-0.75cm);
   %   \draw [gray,-] (-6cm,-2.25cm) -- (6cm,-2.25cm);
   %   \draw [gray,-] (-6cm,-3.75cm) -- (6cm,-3.75cm);
      \node [state] {$x^t$}
      child{ node [state] {$x^{t+1}_\bits{0}$}
        child{ node [state] {$x^{t+2}_{\bits{00}}$}
          child{ node [state] {$x^{t+3}_{\bits{000}}$}}
          child{ node [state] {$x^{t+3}_{\bits{001}}$}}
        }
        child{ node [state] {$x^{t+2}_{\bits{01}}$}
          child{ node [state] {$x^{t+3}_{\bits{010}}$}}
          child{ node [state] {$x^{t+3}_{\bits{011}}$}}
        }
      }
      child{ node [state] {$x^{t+1}_1$}
        child{ node [state] {$x^{t+2}_{\bits{10}}$}
          child{ node [state] {$x^{t+3}_{\bits{100}}$}}
          child{ node [state] {$x^{t+3}_{\bits{101}}$}}
        }
        child{ node [state] {$x^{t+2}_{\bits{11}}$}
          child{ node [state] {$x^{t+3}_{\bits{110}}$}}
          child{ node [state] {$x^{t+3}_{\bits{111}}$}}
        }
      }
      ;
      %\node [state] at (5.5cm, 0) {$u^{t}$};
      %\node [state] at (5.5cm, -1.5) {$u^{t+1}$};
      %\node [state] at (5.5cm, -3) {$u^{t+2}$};
      %\node [state] at (5.5cm, -4.5) {$u^{t+3}$};
    \end{tikzpicture}
  }
  \caption{Metropolis--Hastings conceptualized as a binary tree.
  Nodes at depth~$d$ correspond to iteration~$d$, where the root is at depth~0,
  and branching to the right/left indicates that the proposal is accepted/rejected.
  Each subscript is a sequence, of length~$d$, of~\bits{0}'s and~\bits{1}'s,
  corresponding to the history of rejected and accepted proposals
  with respect to the root.  
  %The random variates (on right) are shared across the layer.}
  }
  \label{tree}
\end{figure}

Na\"{i}ve prefetching can be improved by observing that the two branches are not 
taken with equal probability.
On average, the reject branch tends to be more probable;
the classic result for the optimal MH acceptance rate is 
0.234~\citep{roberts-1997-accept}, so most prefetching scheduling policies
have been built around the expectation of rejection.
Let~${\alpha \le 0.5}$ be the expected acceptance rate. 
\citet{byrd-2008-SMP} introduced \emph{speculative moves}, a procedure that 
speculatively evaluates only along the `reject' branch of the binary tree;
in Figure~(\ref{tree}), this corresponds to the left-most branch.
In each round of their algorithm, only the first~$k$ out of~${J-1}$ extra cores 
perform useful work, where~$k$ is the number of rejected proposals before the
first accepted proposal, relative to the root of the tree.
The expected speedup is then:
\begin{align*}
1 + \E(k) < 1 + \sum_{k=0}^\infty k (1-\alpha)^k \alpha
< 1 + \frac{1-\alpha}{\alpha} = \frac{1}{\alpha}\,.
\end{align*}
The first term on the left is due to the core at the root of the tree, 
which always performs useful computation in prefetching schemes.
For an acceptance rate of ${\alpha=0.23}$, this scheme yields a maximum expected 
speedup of about~$4.3$, reaching about~$4$ with~$16$ cores, and thus is more 
limited than the na\"{i}ve prefetching policy since it essentially cannot take 
advantage of additional cores.
\citet{byrd-2010-speculative} later considered the special case where the
evaluation of the target occurs on two timescales, slow and fast.
This method, called \emph{speculative chains}, modifies speculative moves so
that when the target evaluation is slow, available cores are used to 
speculatively evaluate the subsequent chain, assuming the slow step accepts.

Further extensions to the na\"{i}ve prefetching scheme allocate cores according
to the optimal `tree shape' with respect to various assumptions about the
probability of rejecting a proposal, \ie by greedily allocating cores to nodes
that maximize the depth of speculative computation expected to be 
correct~\citep{strid-2010-prefetching}.
Next, we summarize Strid's schemes and reference related ideas.
\emph{Static prefetching} assumes a fixed acceptance rate; versions of this were
proposed earlier in the context of simulated annealing~\citep{witte-1991-SA}.
\emph{Dynamic prefetching} estimates the acceptance probabilities,
\eg at each level of the tree by drawing empirical MH samples (100,000 in the 
evaluation), or at each  branch in the tree by computing~$\min(\beta, \hat{r})$ 
where~$\beta$ is a constant (${\beta = 1}$ in the evaluation) and~$\hat{r}$ is
an estimate of the MH acceptance ratio based on a fast approximation to the
target function. 
Alternatively, Strid proposes using the approximate target function to identify
the single most likely path on which to perform speculative computation.
Strid also combines prefetching with other sources of parallelism to obtain a 
multiplicative effect.
To the best of our knowledge, these prefetching methods have been
evaluated on up to 64 cores, although usually many fewer.

%In recent other work,~\citet{banterle-2014-delayed} combine prefetching ideas
%with a delayed acceptance MH rule.~\todo{How to cite?}

In the next chapter, we propose~\emph{predictive prefetching},
a new scheme that, like Strid's dynamic prefetching, uses an approximation
to the target function to predict what computations to prefetch.
There are several fundamental differences between our work and Strid's.
Most critically, we model the error of the target density approximation,
and thus the uncertainty of whether a proposal will be accepted.
%This significantly improved our speedup on several problems.
In addition, we identify a broad class of MCMC algorithms that could benefit
from prefetching, not just Metropolis--Hastings, and we show how prefetching
can exploit a series of approximations, not just a single one.

%\section{Two-stage MCMC}
%\cite{murray-2007-thesis}
%\subsection{MCMC with an approximation}
%\cite{christen-fox-2005-approx}

\section{Approximations and large-scale Bayesian inference}

Real-world problems are rarely amenable to exact inference, so they require 
approximate inference in the form of Monte Carlo estimates or
\emph{variational} approximations.
Unfortunately, approximate Bayesian inference can be challenging when modeling
large data sets, as the target posterior density may become expensive to evaluate.  
This challenge has motivated new methods for inferential computation that can
take advantage of approximations to the target density,
most often by examining only a subset of the data,
or by exploiting closed form approximations
such as Taylor series~\citep{christen-fox-2005-approx},
or by fitting linear or Gaussian process regressions~\citep{conrad:2014-local}.

In Bayesian inference, the target density involves a likelihood, which often 
decomposes into a product of many factors corresponding to data items, \eg
\be
\pi(\theta \given \x) = \pi_0(\theta) \pi(\x \given \theta) 
                 = \pi_0(\theta) \prod_{n=1}^{N} \pi(x_n \given \theta).
\label{bayesian}
\ee
Below, we survey MCMC sampling schemes that exploit this factorization property,
motivated by large-scale Bayesian inference with large datasets.

\subsection{Embarrassingly parallel, approximate MCMC}

Several authors have suggested partitioning a large dataset into multiple shards
and running MCMC inference on each partition separately across parallel cores.
Each of $J$ shards $\{\x^{(j)}\}_{j=1}^J$ defines what is sometimes called
a \emph{subposterior}:
\[
\pi^{(j)}(\theta \given \x^{(j)})
= \pi_0(\theta)^{1/J} \prod_{x \in \x^{(j)}} \pi(x \given \theta),
\quad j = 1, \dots, J.
\]
The contribution from the original prior is down-weighted so that
the original posterior is equal to the product of the $J$ subposteriors, \ie
$\pi(\theta \given \x) = \prod_{j=1}^J \pi^{(j)}(\theta \given \x{(j)})$.
However, it is not clear how to combine the samples from the $J$ subposteriors
in a coherent fashion to estimate functions of the desired full posterior.
Below, we describe three recent efforts.

\citet{neiswanger-2013-ep} explore three potential solutions, ranging from
parametric to non-parametric and semi-parametric models.
For example, their parametric model invokes the Bayesian central limit theorem;
they argue that since a posterior looks like a Gaussian in the limit of many
data items, they fit each subposterior with a Gaussian, and then approximate the
full posterior as a product of these approximate subposteriors.

\citet{scott-2013-consensus} propose \emph{consensus Monte Carlo}, 
which combines the subposteriors through a weighted average.
For Gaussian models, the optimal weight of the~$j$th subposterior
is~$W_j = \Sigma_j^{-1}$, the inverse of the covariance matrix~$\Sigma_j$
of the subposterior.
Assuming a Gaussian model, the authors Monte Carlo estimate~$\Sigma_j$ using
the empirical sample variance from the corresponding subposterior.
%An alternate proposal weights each element of $\theta$ by the reciprocal of
%its marginal posterior variance.

Finally, \citet{dunson:2013-weierstrass} propose a \emph{Weierstrass sampler}
for parallel MCMC on independent data partitions; these authors provide analytic
bounds on the approximation error of their sampler, which appears to be more
robust than those described above.

\subsection{MCMC with mini-batches}

Other methods for accelerating MCMC sampling in the case of large-scale
Bayesian inference are inspired by stochastic gradient descent.
Traditional gradient descent performs optimization by iteratively
computing and following a local gradient that depends on a sum of terms
corresponding to data items~\citep{optimization:1983-book}.
Stochastic gradient descent is remarkably simple and effective:
at each iteration, it uses an approximate gradient based on only a
random subset of data, called a \emph{mini-batch}, or even just a single 
datum~\citep{murata:1998-statistical}.
Stochastic variational inference techniques adapt these ideas to 
variational inference~\citep{hoffman:2013-svi}, a class of Bayesian 
procedures that can be efficient but are only approximate in the sense of
lacking MCMC's feature of asymptotic correctness.

With MCMC, the idea is to evaluate an approximate posterior whose likelihood
term is a noisy estimate based on sampling only one or a few data items.
Recent approaches have implemented efficient transition operators that
lead to approximate stationary distributions~\citep{welling-2011-langevin,welling-2012-fisher,korattikara-2014-austerity,bardenet:2014-subsampling,doucet-2014-likelihood-estimator}.
Other recent work uses a lower bound on the local likelihood factor to simulate
from the exact posterior distribution while evaluating only a subset of the data
at each iteration~\citep{maclaurin-2014-firefly}.

%Ensuring that these estimates have the proper scale requires some care.
%ore generally, is not obvious whether MCMC procedures tolerate this
%approach, and a critical component of this research is to understand
%performance in terms of accuracy, as a function of the subsample size.

%This analysis will likely be related to recent results about
%``exact approximate'' pseudo-marginal MCMC, which
%uses a Monte Carlo estimate of the likelihood term~\cite{andrieu:2009-pseudo}
%and was originally developed for a genetics application using importance
%sampling for the estimate~\cite{beaumont:2003-is}.

%recent developments in Markov chain Monte Carlo (MCMC) have implemented efficient transition %operators that
%lead to approximate stationary distributions~\citep{Welling2011, Korattikara2014, Bardenet2014}.
%
%Recent other work uses a lower bound on the local likelihood factor to simulate from the exact posterior distribution while evaluating only a subset of the data at each iteration~\citep{maclaurin-2014-firefly}.

%\section{Other methods that may end up in this chapter}
%Adaptive methods (depend on the history of the chain) \\
%pseudo-marginal (MC estimate the target) \\
%austere MCMC \\
%(MC$)^3$ (something parallel) \\
%coupling from the past (perfect estimators) \\
%Langevin dynamics (cross between SGD and MCMC) \\
%online MCMC (there seems to be one attempt at this)

\bigskip

In the rest of this thesis, we focus on accelerating MCMC by combining
parallelism with approximations to the transition operator through
prefetching ideas.
Notably, we arrive at a method in which the stationary distribution is 
\emph{exactly} the target posterior.

%We attack the problem using parallelism.

\end{document}